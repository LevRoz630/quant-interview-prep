\documentclass[11pt]{article}

% ---------- Layout ----------
\usepackage[margin=0.75in]{geometry}
\usepackage{parskip}

% ---------- Math ----------
\usepackage{amsmath, amssymb}

% ---------- Lists / tables ----------
\usepackage{enumitem}
\usepackage{booktabs}

% ---------- Links ----------
\usepackage[hidelinks]{hyperref}

% ---------- Encoding ----------
\usepackage[T1]{fontenc}
\usepackage[utf8]{inputenc}

% ---------- Colors ----------
\usepackage{xcolor}
\definecolor{darkblue}{rgb}{0.0, 0.0, 0.5}
\definecolor{darkgreen}{rgb}{0.0, 0.5, 0.0}

% ---------- Custom commands ----------
\newcommand{\example}[1]{\textcolor{darkblue}{\textbf{Example:}} #1}
\newcommand{\technique}[1]{\textcolor{darkgreen}{\textbf{#1}}}

% ---------- Title ----------
\title{\textbf{Mental Math Techniques}\\Fast Arithmetic for Trading Interviews}
\author{}
\date{}

\begin{document}
\maketitle
\tableofcontents
\newpage

\section{Core Philosophy}

\textbf{Key principle:} Use standard decompositions and patterns, not "normal" arithmetic.

\begin{itemize}[leftmargin=*]
\item \textbf{Reduce cognitive load} --- apply patterns automatically
\item \textbf{Avoid carrying/borrowing} --- use complements and decompositions
\item \textbf{Round then adjust} --- work with easy numbers first
\item \textbf{Pair to nice numbers} --- look for sums to 10, 100, 1000
\item \textbf{Practice until automatic} --- these must be reflexive under pressure
\end{itemize}

\section{Multiplication Techniques}

\subsection*{1. Two-digit $\times$ two-digit: Decompose intelligently}

\textbf{General formula:} $(a+b)(c+d)=ac+ad+bc+bd$

But choose $b$ and $d$ to be easy (typically multiples of 10).

\vspace{0.5cm}
\example{$47\times 63$}

Rewrite as $47\times(70-7)$:
\begin{align*}
47\times 70 &= 3290\\
47\times 7 &= 329\\
47\times 63 &= 3290-329=\boxed{2961}
\end{align*}

\vspace{0.5cm}
\example{$38\times 24$}

Rewrite as $(40-2)\times 24$:
\begin{align*}
40\times 24 &= 960\\
2\times 24 &= 48\\
38\times 24 &= 960-48=\boxed{912}
\end{align*}

\vspace{0.5cm}
\example{$56\times 17$}

Rewrite as $56\times(20-3)$:
\begin{align*}
56\times 20 &= 1120\\
56\times 3 &= 168\\
56\times 17 &= 1120-168=\boxed{952}
\end{align*}

\subsection*{2. Numbers near 100}

\textbf{Formula:} $(100-x)(100-y)=10000-100(x+y)+xy$

This is extremely fast because you only work with small numbers $x$ and $y$.

\vspace{0.5cm}
\example{$97\times 94$}

Here $x=3$, $y=6$:
\begin{align*}
10000-100(3+6)+3\times 6 &= 10000-900+18\\
&= \boxed{9118}
\end{align*}

\vspace{0.5cm}
\example{$98\times 96$}

Here $x=2$, $y=4$:
\begin{align*}
10000-100(2+4)+2\times 4 &= 10000-600+8\\
&= \boxed{9408}
\end{align*}

\vspace{0.5cm}
\example{$93\times 91$}

Here $x=7$, $y=9$:
\begin{align*}
10000-100(7+9)+7\times 9 &= 10000-1600+63\\
&= \boxed{8463}
\end{align*}

\subsection*{3. Numbers near 50}

\textbf{Formula:} $(50+x)(50+y)=2500+50(x+y)+xy$

\vspace{0.5cm}
\example{$53\times 47$}

Here $x=3$, $y=-3$:
\begin{align*}
2500+50(3-3)+3\times(-3) &= 2500+0-9\\
&= \boxed{2491}
\end{align*}

\vspace{0.5cm}
\example{$56\times 54$}

Here $x=6$, $y=4$:
\begin{align*}
2500+50(6+4)+6\times 4 &= 2500+500+24\\
&= \boxed{3024}
\end{align*}

\vspace{0.5cm}
\example{$48\times 52$}

Here $x=-2$, $y=2$:
\begin{align*}
2500+50(-2+2)+(-2)\times 2 &= 2500+0-4\\
&= \boxed{2496}
\end{align*}

\subsection*{4. Squaring numbers ending in 5}

\textbf{Formula:} $(10a+5)^2=100a(a+1)+25$

This is blazing fast.

\vspace{0.5cm}
\example{$35^2$}

Here $a=3$:
\[
100\times 3\times 4+25=1200+25=\boxed{1225}
\]

\vspace{0.5cm}
\example{$65^2$}

Here $a=6$:
\[
100\times 6\times 7+25=4200+25=\boxed{4225}
\]

\vspace{0.5cm}
\example{$85^2$}

Here $a=8$:
\[
100\times 8\times 9+25=7200+25=\boxed{7225}
\]

\vspace{0.5cm}
\example{$125^2$}

Here $a=12$:
\[
100\times 12\times 13+25=15600+25=\boxed{15625}
\]

\subsection*{5. Multiply by 11}

\textbf{Pattern:} $\overline{ab}\times 11=\overline{a(a+b)b}$ (carry if $a+b\ge 10$)

\vspace{0.5cm}
\example{$53\times 11$}

Middle digit: $5+3=8$
\[
53\times 11=\boxed{583}
\]

\vspace{0.5cm}
\example{$67\times 11$}

Middle digit: $6+7=13$ (carry 1)
\[
67\times 11=(6+1)\,3\,7=\boxed{737}
\]

\vspace{0.5cm}
\example{$84\times 11$}

Middle digit: $8+4=12$ (carry 1)
\[
84\times 11=(8+1)\,2\,4=\boxed{924}
\]

\vspace{0.5cm}
\textbf{Three-digit:} $\overline{abc}\times 11=\overline{a(a+b)(b+c)c}$

\example{$234\times 11$}
\[
234\times 11=2\,(2+3)\,(3+4)\,4=2574
\]
But $3+4=7$, so: $\boxed{2574}$

\example{$578\times 11$}

Digits: $5$, $(5+7)=12$ (carry), $(7+8)=15$ (carry), $8$
\[
578\times 11=5\,(12)\,(15)\,8\to 5\,(13)\,5\,8\to 6\,3\,5\,8=\boxed{6358}
\]

\subsection*{6. Multiply by 5, 25, 125}

\technique{Multiply by 5:} $\times 10\div 2$

\example{$87\times 5=870\div 2=\boxed{435}$}

\example{$124\times 5=1240\div 2=\boxed{620}$}

\vspace{0.5cm}
\technique{Multiply by 25:} $\times 100\div 4$

\example{$36\times 25=3600\div 4=\boxed{900}$}

\example{$84\times 25=8400\div 4=\boxed{2100}$}

\vspace{0.5cm}
\technique{Multiply by 125:} $\times 1000\div 8$

\example{$16\times 125=16000\div 8=\boxed{2000}$}

\example{$72\times 125=72000\div 8=\boxed{9000}$}

\subsection*{7. Divide by 5, 25}

\technique{Divide by 5:} $\times 2\div 10$

\textbf{This is huge!} Dividing by 5 is hard; multiplying by 2 is trivial.

\example{$347\div 5=694\div 10=\boxed{69.4}$}

\example{$825\div 5=1650\div 10=\boxed{165}$}

\vspace{0.5cm}
\technique{Divide by 25:} $\times 4\div 100$

\example{$675\div 25=2700\div 100=\boxed{27}$}

\example{$1250\div 25=5000\div 100=\boxed{50}$}

\subsection*{8. Approximation + correction}

\textbf{Strategy:} Compute via an easy nearby number, then adjust.

This is safer than direct multiplication under pressure.

\vspace{0.5cm}
\example{$79\times 46$}

Use $80\times 46$:
\begin{align*}
80\times 46 &= 3680\\
\text{subtract } 46\colon\quad 79\times 46 &= 3680-46=\boxed{3634}
\end{align*}

\vspace{0.5cm}
\example{$68\times 29$}

Use $68\times 30$:
\begin{align*}
68\times 30 &= 2040\\
\text{subtract } 68\colon\quad 68\times 29 &= 2040-68=\boxed{1972}
\end{align*}

\vspace{0.5cm}
\example{$103\times 52$}

Use $100\times 52$:
\begin{align*}
100\times 52 &= 5200\\
\text{add } 3\times 52\colon\quad 103\times 52 &= 5200+156=\boxed{5356}
\end{align*}

\section{Subtraction: Difference Method}

\textbf{Never borrow!} Instead, count up from the smaller number to the larger.

\vspace{0.5cm}
\example{$10000-5873$}

Ask: $5873+?=10000$

Count up:
\begin{align*}
5873 &\to 6000 \quad (+127)\\
6000 &\to 10000 \quad (+4000)\\
\text{Total:}\quad &\boxed{4127}
\end{align*}

\vspace{0.5cm}
\example{$8000-3456$}

Count up:
\begin{align*}
3456 &\to 3500 \quad (+44)\\
3500 &\to 8000 \quad (+4500)\\
\text{Total:}\quad &\boxed{4544}
\end{align*}

\vspace{0.5cm}
\example{$5000-2738$}

Count up:
\begin{align*}
2738 &\to 2800 \quad (+62)\\
2800 &\to 5000 \quad (+2200)\\
\text{Total:}\quad &\boxed{2262}
\end{align*}

\vspace{0.5cm}
\example{$1000-647$}

Count up:
\begin{align*}
647 &\to 650 \quad (+3)\\
650 &\to 1000 \quad (+350)\\
\text{Total:}\quad &\boxed{353}
\end{align*}

This is \textbf{faster and more accurate} under pressure than borrowing.

\section{Addition: Chunking Strategy}

\textbf{Always scan for complements to 10, 100, 1000} before adding left-to-right.

\vspace{0.5cm}
\example{$487+596+213+404$}

Pair to 1000s:
\begin{align*}
(487+\underline{513}) &= 1000 \quad \text{but we have 596, so:}\\
487+596 &= 1083\\
213+404 &= 617\\
\text{Total:}\quad 1083+617 &= \boxed{1700}
\end{align*}

Better approach:
\begin{align*}
(596+404) &= 1000\\
(487+213) &= 700\\
\text{Total:}\quad &= \boxed{1700}
\end{align*}

\vspace{0.5cm}
\example{$38+67+62+33$}

Pair to 100:
\begin{align*}
(38+62) &= 100\\
(67+33) &= 100\\
\text{Total:}\quad &= \boxed{200}
\end{align*}

\vspace{0.5cm}
\example{$145+387+255+613$}

Pair smartly:
\begin{align*}
(145+255) &= 400\\
(387+613) &= 1000\\
\text{Total:}\quad &= \boxed{1400}
\end{align*}

\vspace{0.5cm}
\example{$19+47+81+53$}

Pair to even 100s:
\begin{align*}
(19+81) &= 100\\
(47+53) &= 100\\
\text{Total:}\quad &= \boxed{200}
\end{align*}

\section{Division Shortcuts}

\subsection*{1. Dividing by 9}

\textbf{Pattern:} $n\div 9\approx n\times 0.111\ldots$

Better: use the fact that $9\times 11=99\approx 100$

\example{$456\div 9$}

Think: $456\div 9=456\times\frac{11}{99}=\frac{5016}{99}\approx\frac{5000}{100}=50$

Exact: $456=9\times 50+6$, so $456\div 9=50.666\ldots=\boxed{50\tfrac{2}{3}}$

\subsection*{2. Dividing by 15}

\textbf{Strategy:} $\div 15=\div 3\div 5=\div 3\times 2\div 10$

\example{$450\div 15$}
\begin{align*}
450\div 3 &= 150\\
150\div 5 &= 150\times 2\div 10=30\\
\text{Answer:}\quad &\boxed{30}
\end{align*}

\subsection*{3. Dividing by 12}

\textbf{Strategy:} $\div 12=\div 4\div 3$ or $\div 3\div 4$

\example{$288\div 12$}
\begin{align*}
288\div 4 &= 72\\
72\div 3 &= 24\\
\text{Answer:}\quad &\boxed{24}
\end{align*}

\section{Percentages}

\subsection*{Quick percentage calculations}

\textbf{Key insight:} $a\%$ of $b$ = $b\%$ of $a$

\example{$16\%$ of $25$}

Instead compute $25\%$ of $16=4$

So $16\%$ of $25=\boxed{4}$

\vspace{0.5cm}
\example{$8\%$ of $75$}

Instead compute $75\%$ of $8=6$

So $8\%$ of $75=\boxed{6}$

\vspace{0.5cm}
\example{$12\%$ of $50$}

$12\%=\frac{12}{100}$, so $12\%$ of $50=\frac{12\times 50}{100}=\frac{600}{100}=\boxed{6}$

Or: $50\%$ of $12=6$

\subsection*{Common percentages to memorize}

\begin{center}
\begin{tabular}{ll}
\toprule
\textbf{Percentage} & \textbf{Fraction/Decimal}\\
\midrule
$10\%$ & $0.1$\\
$12.5\%$ & $\tfrac{1}{8}=0.125$\\
$16.666\ldots\%$ & $\tfrac{1}{6}\approx 0.167$\\
$20\%$ & $\tfrac{1}{5}=0.2$\\
$25\%$ & $\tfrac{1}{4}=0.25$\\
$33.333\ldots\%$ & $\tfrac{1}{3}\approx 0.333$\\
$37.5\%$ & $\tfrac{3}{8}=0.375$\\
$40\%$ & $\tfrac{2}{5}=0.4$\\
$50\%$ & $\tfrac{1}{2}=0.5$\\
$62.5\%$ & $\tfrac{5}{8}=0.625$\\
$66.666\ldots\%$ & $\tfrac{2}{3}\approx 0.667$\\
$75\%$ & $\tfrac{3}{4}=0.75$\\
$80\%$ & $\tfrac{4}{5}=0.8$\\
$87.5\%$ & $\tfrac{7}{8}=0.875$\\
\bottomrule
\end{tabular}
\end{center}

\section{Fraction Arithmetic}

\subsection*{Adding fractions with different denominators}

\textbf{Strategy:} Cross-multiply for numerator, multiply denominators

\[
\frac{a}{b}+\frac{c}{d}=\frac{ad+bc}{bd}
\]

\example{$\frac{2}{3}+\frac{3}{5}$}

\[
\frac{2\times 5+3\times 3}{3\times 5}=\frac{10+9}{15}=\frac{19}{15}=1\tfrac{4}{15}
\]

\subsection*{Multiplying fractions}

\textbf{Cancel before multiplying!}

\example{$\frac{15}{28}\times\frac{14}{25}$}

Cancel: $\frac{15}{25}=\frac{3}{5}$ and $\frac{14}{28}=\frac{1}{2}$

\[
\frac{15}{28}\times\frac{14}{25}=\frac{3}{5}\times\frac{1}{2}=\frac{3}{10}
\]

\subsection*{Dividing fractions}

\textbf{Multiply by reciprocal:} $\frac{a}{b}\div\frac{c}{d}=\frac{a}{b}\times\frac{d}{c}=\frac{ad}{bc}$

\example{$\frac{5}{8}\div\frac{3}{4}$}

\[
\frac{5}{8}\times\frac{4}{3}=\frac{5\times 4}{8\times 3}=\frac{20}{24}=\frac{5}{6}
\]

\section{Powers and Roots}

\subsection*{Powers of 2 (memorize these!)}

\begin{center}
\begin{tabular}{ll}
\toprule
\textbf{Power} & \textbf{Value}\\
\midrule
$2^5$ & $32$\\
$2^6$ & $64$\\
$2^7$ & $128$\\
$2^8$ & $256$\\
$2^9$ & $512$\\
$2^{10}$ & $1024$\\
$2^{11}$ & $2048$\\
$2^{12}$ & $4096$\\
$2^{15}$ & $32768$\\
$2^{16}$ & $65536$\\
$2^{20}$ & $1048576\approx 10^6$\\
\bottomrule
\end{tabular}
\end{center}

\subsection*{Square roots via Newton's method (one iteration)}

\textbf{Formula:} $\sqrt{n}\approx\frac{1}{2}\left(x+\frac{n}{x}\right)$ where $x$ is initial guess

\example{$\sqrt{50}$}

Guess $x=7$ (since $7^2=49$):
\[
\sqrt{50}\approx\frac{1}{2}\left(7+\frac{50}{7}\right)=\frac{1}{2}(7+7.14)=\boxed{7.07}
\]
True value: $7.071\ldots$

\example{$\sqrt{80}$}

Guess $x=9$ (since $9^2=81$):
\[
\sqrt{80}\approx\frac{1}{2}\left(9+\frac{80}{9}\right)=\frac{1}{2}(9+8.89)=\boxed{8.94}
\]
True value: $8.944\ldots$

\section{Practice Drills}

\subsection*{Multiplication drills (do these daily)}

\begin{enumerate}[leftmargin=*]
\item $47\times 23$
\item $96\times 98$
\item $53\times 47$
\item $125^2$
\item $67\times 11$
\item $84\times 25$
\item $78\times 19$
\item $94\times 96$
\end{enumerate}

\subsection*{Subtraction drills}

\begin{enumerate}[leftmargin=*]
\item $10000-6842$
\item $5000-2963$
\item $8000-4157$
\item $7500-3278$
\end{enumerate}

\subsection*{Addition drills (look for pairs)}

\begin{enumerate}[leftmargin=*]
\item $237+763+491+509$
\item $88+67+12+33$
\item $456+789+544+211$
\end{enumerate}

\subsection*{Division drills}

\begin{enumerate}[leftmargin=*]
\item $735\div 5$
\item $1250\div 25$
\item $456\div 12$
\item $675\div 15$
\end{enumerate}

\section{Interview Strategy}

\subsection*{During mental math sections}

\begin{itemize}[leftmargin=*]
\item \textbf{Talk through your method} --- let them see your process
\item \textbf{Use patterns, not brute force} --- show you have systematic techniques
\item \textbf{Round and adjust} --- demonstrate strategic thinking
\item \textbf{Write intermediate steps} --- reduces errors, shows clarity
\item \textbf{Check reasonableness} --- does $97\times 94\approx 9000$? Yes.
\item \textbf{Practice under time pressure} --- set 30-second timers
\end{itemize}

\subsection*{What interviewers are evaluating}

\begin{enumerate}[leftmargin=*]
\item \textbf{Speed} --- Can you compute quickly?
\item \textbf{Accuracy} --- Do you make careless errors?
\item \textbf{Method} --- Do you use smart techniques or struggle?
\item \textbf{Composure} --- Do you panic or stay calm under pressure?
\end{enumerate}

\vspace{1cm}
\begin{center}
\textit{These techniques are \textbf{learnable skills}, not innate talent.}\\
\textit{Practice 15 minutes daily for 2 weeks and they become automatic.}
\end{center}

\end{document}
